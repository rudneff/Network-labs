\documentclass[a4paper,12pt]{article}

\input{header.tex}

\title{Отчёт по лабораторной работе \\ <<Динамическая IP-маршрутизация>>}
\author{Руднев Дмитрий Николаевич}

\begin{document}

\maketitle

\tableofcontents

\section{Настройка сети}

\subsection{Топология сети}

Топология сети и используемые IP-адреса показаны на рисунке~\ref{fig:network}.

\begin{figure}
\centering
\includegraphics[width=0.8\textwidth]{includes/network_gv.pdf}
\caption{Топология сети}
\label{fig:network}
\end{figure}

Перечень узлов, на которых используется динамическая IP-маршрутизация: r1, r2, r3, r4, r5


\subsection{Назначение IP-адресов}

Ниже приведён файл сетевой настройки  маршрутизатора r1.

\begin{Verbatim}
auto lo
iface lo inet loopback

auto eth0
iface eth0 inet static
address 10.0.20.1
netmask 255.255.255.0

auto eth1
iface eth1 inet static
address 10.0.10.1
netmask 255.255.255.0
\end{Verbatim}

Ниже приведён файл сетевой настройки рабочей станции wsp1.

\begin{Verbatim}
auto lo
iface lo inet loopback

auto eth0
iface eth0 inet static
address 10.0.10.3
netmask 255.255.255.0
gateway 10.0.10.1
\end{Verbatim}



\subsection{Настройка протокола RIP}

Ниже приведен файл \Code{/etc/quagga/ripd.conf} маршрутизатора r1.

\begin{Verbatim}
! Этот настройки, касающиеся протокола RIP.
router rip

! Раскомментируйте ниже все интерфейсы, подключённые
! к сетям с другими маршрутизаторами.
network eth0
network eth1
! network eth2

! Уменьшаем значения всех таймеров для ускорения опытов.
! Рассылка: 10 сек., устаревание: 60 cек., сборка мусора: 120 сек.
timers basic 10 60 120

! Следующие две строчки заставляют маршрутизатор
! добавлять в сообщения протокола RIP все известные ему маршруты.
redistribute kernel
!redistribute connected

! Это имя файла журнала службы RIP.
! Его содержимое можно изучить в случае неполадок
log file /var/log/quagga/ripd.log
\end{Verbatim}



Ниже приведен файл \Code{/etc/quagga/ripd.conf} рабочий станции, связанной с несколькими маршрутизаторами (указать, какой).

\begin{Verbatim}
Такой нет
\end{Verbatim}


\section{Проверка настройки протокола RIP}

Вывод \textbf{traceroute} от узла r4 до wsp1 при нормальной работе сети.

\begin{Verbatim}
r4:~# traceroute 10.0.10.3
traceroute to 10.0.10.3 (10.0.10.3), 64 hops max, 40 byte packets
 1  10.0.30.1 (10.0.30.1)  8 ms  0 ms  0 ms
 2  10.0.10.3 (10.0.10.3)  31 ms  1 ms  1 ms
\end{Verbatim}

Вывод \textbf{traceroute} от узла такого-т до внешнего IP (195.19.38.2 сгодится).

\begin{Verbatim}
Сюда нужно поместить вывод traceroute.
\end{Verbatim}

Вывод сообщения RIP.

\begin{Verbatim}
Перехваченное сообщение RIP от любого маршрутизатора
\end{Verbatim}

Вывод таблицы RIP.

\begin{Verbatim}
Таблица RIP
\end{Verbatim}

Вывод таблицы маршрутизации.

\begin{Verbatim}
Таблица маршрутизации
\end{Verbatim}

\section{Расщепленный горизонт и испорченные обратные обновления}

Поместить сюда вывод сообщения одного и того же маршрутизатор с включенным расщ. горизонтом, с включенными испорченными обновлениями, с отключённым расщ. гор.

Объяснить разницу.

Вернуть настройки в исходное состояние (включенный без испорченных).

\section{Имитация устранимой поломки в сети}

Какой маршрутизатор выключили?

Вывод таблицы RIP непосредственно перед истечением таймера устаревания (на маршрутизаторе-соседе отключенного).

\begin{Verbatim}
Таблица RIP
\end{Verbatim}

Перестроенная таблица на этом же маршрутизаторе

\begin{Verbatim}
Таблица RIP
\end{Verbatim}


Вывод \textbf{traceroute} от узла такого-то до такого-то после того, как служба RIP перестроила таблицы маршрутизации.

\begin{Verbatim}
Сюда нужно поместить вывод traceroute после "поломки".
\end{Verbatim}

\section{Имитация неустранимой поломки в сети}

Какой маршрутизатор выключили? (Теперь у нас нет связанной сети)

Далее поместить таблицы протокола RIP, где видна 16-ая метрика, и сообщения протокола RIP с 16-ой метрикой.

\end{document}
